%%=============================================================================
%% Inleiding
%%=============================================================================

\chapter{Inleiding}
\label{ch:inleiding}

Deze bachelorproef was oorspronkelijk bedoeld om fraudegevoeligheid in bedrijven
te onderzoeken en een oplossing te presenteren voor bepaalde problemen. Dit is
echter een gevoelig en moeilijk te onderzoeken onderwerp. Bedrijven houden
namelijk hun zwaktes en problemen intern \autocite{EconomicCrimeSurvey}. Door
deze begrijpbare moeilijheid is het onderwerp aangepast naar hoe het bewijzen en
verzekeren van \gls{authenticiteit}, \gls{integriteit} en \gls{toerekenbaarheid}
bedrijfsprocessen
kan beveiligen.

Er word in deze bachelorproef gekeken naar hoe deze zaken kunnen verzekerd
worden en welke protocollen of tools hiervoor in aanmerking komen. Specifiek
word er vergeleken of derde partijen deze dienst betrouwbaar kunnen regelen
tegenover eigen systemen en decentrale systemen.

\section{Stand van zaken}
\label{sec:stand-van-zaken}
Er zijn geen bestaande onderzoeken relevant aan deze bachelorproef. Deze
bachelorproef kan echter dienen als een basis voor een nieuw onderzoek dat
eventueel de oplossing die beschreven is in deze bachelorproef, uittest.

%% TODO: deze sectie (die je kan opsplitsen in verschillende secties) bevat je
%% literatuurstudie. Vergeet niet telkens je bronnen te vermelden!


\section{Probleemstelling en Onderzoeksvragen}
\label{sec:onderzoeksvragen}

Deze bachelorproef probeert te beantwoorden hoe het verzekeren van
\gls{authenticiteit}, \gls{integriteit} en \gls{toerekenbaarheid} bedrijven
kunnen helpen om
bepaalde processen te beveiligen. Er word onderzoek gedaan naar wat
\gls{authenticiteit}, \gls{integriteit} en \gls{toerekenbaarheid} kan verzekeren
en naar welke
processen of problemen waarin dit kan gebruikt worden.

Verder wordt er nagekeken hoe deze oplossingen kunnen geïntegreerd worden in een
bedrijf. Hierbij wordt er gefocust op configuratie van de oplossing en het
contact dat medewerkers met het systeem zullen hebben.

De bedoeling is dat deze bachelorproef kan gebruikt worden om de oplossing
veilig te implementeren.
%% TODO:
%% Uit je probleemstelling moet duidelijk zijn dat je onderzoek een meerwaarde
%% heeft voor een concrete doelgroep (bv. een bedrijf).
%%
%% Wees zo concreet mogelijk bij het formuleren van je
%% onderzoeksvra(a)g(en). Een onderzoeksvraag is trouwens iets waar nog
%% niemand op dit moment een antwoord heeft (voor zover je kan nagaan).

\section{Opzet van deze bachelorproef}
\label{sec:opzet-bachelorproef}

%% TODO: Het is gebruikelijk aan het einde van de inleiding een overzicht te
%% geven van de opbouw van de rest van de tekst. Deze sectie bevat al een aanzet
%% die je kan aanvullen/aanpassen in functie van je eigen tekst.

De rest van deze bachelorproef is als volgt opgebouwd:

In Hoofdstuk~\ref{ch:methodologie} wordt de methodologie toegelicht en worden de
gebruikte onderzoekstechnieken besproken om een antwoord te kunnen formuleren op
de onderzoeksvragen.

%% TODO: Vul hier aan voor je eigen hoofstukken, één of twee zinnen per
%hoofdstuk
In Hoofdstuk~\ref{ch:voorbeelden-van-problemen} worden bepaalde problemen in
bedrijven aangekaart.

In Hoofdstuk~\ref{ch:identiteit-bewijzen} wordt onderzocht welke systemen en
protocollen \gls{authenticiteit}, \gls{integriteit} en \gls{toerekenbaarheid}
kunnen bewijzen

In Hoofdstuk~\ref{ch:certificaatautoriteiten} worden certicaatautoriteiten en
hun
voor- en nadelen onderzocht.

In Hoofdstuk~\ref{ch:web-of-trust} wordt een Web Of Trust en de voor- en
nadelen hiervan onderzocht.

In Hoofdstuk~\ref{ch:implementatie-van-een-gpg-web-of-trust} wordt een
implementatie van een \gls{GPG} Web Of Trust onderzocht. De verschillende mogelijke
opties worden vergeleken en beargumenteerd.

In Hoofdstuk~\ref{ch:fysieke-beveiliging-met-een-wot} wordt gekeken hoe fysieke
beveiliging kan geïmplementeert worden met een Web Of Trust. Het bevat een link
tussen de digitale en de fysieke wereld.

In Hoofdstuk~\ref{ch:conclusie}, tenslotte, wordt de conclusie gegeven en een
antwoord geformuleerd op de onderzoeksvragen. Daarbij wordt ook een aanzet
gegeven voor toekomstig onderzoek binnen dit domein.
