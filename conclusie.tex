%%=============================================================================
%% Conclusie
%%=============================================================================

\chapter{Conclusie}
\label{ch:conclusie}

%% TODO: Trek een duidelijke conclusie, in de vorm van een antwoord op de
%% onderzoeksvra(a)g(en). Wat was jouw bijdrage aan het onderzoeksdomein en
%% hoe biedt dit meerwaarde aan het vakgebied/doelgroep? Reflecteer kritisch
%% over het resultaat. Had je deze uitkomst verwacht? Zijn er zaken die nog
%% niet duidelijk zijn? Heeft het onderzoek geleid tot nieuwe vragen die
%% uitnodigen tot verder onderzoek?

Er kan geconcludeerd worden dat een eigen implementatie van een Web Of Trust kan
gebruikt worden om bedrijfsprocessen te beveiligen. Het is echter zeer
belangrijk dat zo'n systeem juist wordt geïmplementeerd.

Een vraag die onbeantwoord is, is hoe data succesvol kan beveiligd worden door
korte of zwakke wachtwoorden. Zou een \textit{lock out} kunnen geïmplementeerd
worden in \gls{GPG}? Het antwoord zou kunnen gebruik maken van een systeem
waarin de kost van een enkele gok naar het wachtwoord te hoog ligt.

Cryptografie zal over de loop van de jaren veranderen, het is daarom verwacht
van de lezer om rekening te houden met de datum waarop deze bachelorproef is
geschreven wannner zaken zoals bit grootte worden besproken. De algoritmen voor
assymetrische encryptie zullen veranderen. \textit{Elliptic curve cryptography}
zal hoogstwaarschijnlijk RSA vervangen. Een eventueele doorbraak van quantum
computing kan het einde betekenen van alle vormen van assymetrische encryptie.
Daarom is het belangrijk, zoals altijd in de IT-wereld, om evolutie in het oog
te houden. Post quantum cryptografie wordt momenteel al onderzocht
\autocite{ANewHopeUsenix}.

Toekomstige ontwikkeling zou eventueel een nieuw decentraal systeem introduceren
of misschien zelf een centraal systeem dat volledig te vertrouwen valt.

Wat betreft de fysieke beveiliging zou het veiliger zijn indien de badge enkel
kon uitgelezen worden door een vertrouwde kaartlezer. Hierbij zou een kaartlezer
zelf ook een sleutelpaar hebben dat ondertekent is door de private sleutel van
de \textit{access control} server. Dit vereist echter dat de badge zelf
berekeningen kan doen. Specifiek zou de badge handtekeningen moeten kunnen
verifiëren. Mogelijks zouden smartphones met NFC technologie hierbij kunnen
helpen.
