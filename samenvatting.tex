%%=============================================================================
%% Samenvatting
%%=============================================================================

%% TODO: De "abstract" of samenvatting is een kernachtige (~ 1 blz. voor een
%% thesis) synthese van het document.
%%
%% Deze aspecten moeten zeker aan bod komen:
%% - Context: waarom is dit werk belangrijk?
%% - Nood: waarom moest dit onderzocht worden?
%% - Taak: wat heb je precies gedaan?
%% - Object: wat staat in dit document geschreven?
%% - Resultaat: wat was het resultaat?
%% - Conclusie: wat is/zijn de belangrijkste conclusie(s)?
%% - Perspectief: blijven er nog vragen open die in de toekomst nog kunnen
%%    onderzocht worden? Wat is een mogelijk vervolg voor jouw onderzoek?
%%
%% LET OP! Een samenvatting is GEEN voorwoord!

\chapter*{\IfLanguageName{dutch}{Samenvatting}{Abstract}}

Deze bachelorproef onderzoekt algemene problemen en fraude in bedrijfsprocessen
aan de hand van eerste hand ervaringen en beschrijft hoe deze kunnen opgelost
worden. Het doel is om bedrijven ten eerste duidelijk te maken welke problemen
er zijn en, ten tweede, om een oplossing te vinden. Ten laatste wordt behandeld
hoe de gevonden oplossing, namelijk een \acrlong{WOT}, kan geïmplementeerd
worden.

Bepaalde problemen die vermeld zijn nemen toe qua aantal voorvallen. Dit leidt
tot zowel economische schade als schade ten op zichte van de naam van de
firma. Bedrijven zouden deze bachelorproef als basis kunnen gebruiken om een
systeem uit te rollen dat de vermelde problemen veiligstelt.

Er wordt in deze bachelorproef gezocht naar methodes die \gls{authenticiteit},
\gls{integriteit} en \gls{toerekenbaarheid} verzekeren en deze methodes worden
vervolgens vergeleken. De vergelijking concludeert dat een \acrlong{WOT} een
betere methode is voor het huidige doel en vervolgens wordt het \acrlong{WOT}
onder de loep genomen en geïmplementeerd.

Deze bachelorproef zou kunnen opgevolgd worden door een \textit{proof of
concept} of volledige implementatie dat de werking van de gegeven oplossing zou
bewijzen.
