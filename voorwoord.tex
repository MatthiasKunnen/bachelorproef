%%=============================================================================
%% Voorwoord
%%=============================================================================

\chapter*{Voorwoord}
\label{ch:voorwoord}

%% TODO:
%% Het voorwoord is het enige deel van de bachelorproef waar je vanuit je
%% eigen standpunt (``ik-vorm'') mag schrijven. Je kan hier bv. motiveren
%% waarom jij het onderwerp wil bespreken.
%% Vergeet ook niet te bedanken wie je geholpen/gesteund/... heeft

Deze bachelorproef was voor mij een mogelijkheid om mij te verdiepen in de
complexiteit en omvang van cryptografie en de toepassing hiervan in bedrijven.
Het was mijn doel om de mogelijkheden van cryptografie te onderzoeken en, aan de
hand van bewezen feiten, toepassingen hiervan bloot te leggen.

De vele gesprekken en discussies met zowel experts in het veld als potentiële
gebruikers omtrent het gemak in gebruik in een toepassing als de veiligheid
ervan waren een zeer interessante tijdsbesteding.

\section*{Met speciale dank aan}
Ik zou graag Stefaan Truijen bedanken voor zijn toespraak in HoGent met als
titel \textit{Crypotographically securing habits}. Deze bachelorproef heeft als
basis vele zaken gebruikt die tijdens deze lezing vermeld geweest zijn. Stefaan
en ik hebben verder gesproken over cryptografie op de Gears conferentie. Zijn
goedkeuring voor de toepassingen van een Web Of Trust die ik in mijn
bachelorproef aankaart waren een welkom signaal dat ik juist zat. Ook zijn help
met het onderwerp \textit{\nameref{ch:fysieke-beveiliging-met-een-wot}}  was
zeer geapprecieerd.

Verder heeft \textcite{TruijenStefaan} hoofdstuk
\fullref{ch:implementatie-van-een-gpg-web-of-trust} en hoofdstuk
\fullref{ch:fysieke-beveiliging-met-een-wot} beoordeeld. Zijn opmerkingen
hierover zijn terug te vinden in bijlage (sectie
\fullref{sec:opmerkingen-cryptografisch-expert}). Na zijn beoordeling zijn
natuurlijk gepaste veranderingen gemaakt.

Ook wil ik graag Tom Van den Bulcke bedanken voor het vervullen van de rol van
co-promotor. Zijn voorstellen en verificatie van technische aspecten van deze
bachelorproef waren zeer welkom.

Daarnaast wil ik Jonas Kunnen bedanken voor zijn hulp met elektrische schema's
en componenten en de vele discussies over deze bachelorproef.

Ik zou ook mijn promotor, Marc Asselberg, willen bedanken voor zijn snelle
antwoorden op mijn vragen.

Anderen die ik wil bedanken: Elien Callens, Henri Jacobs en Kathleen Podevyn.
