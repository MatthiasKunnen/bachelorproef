\newglossaryentry{hash}
{
	name={hash},
	description={
		Een hash functie is een functie die een tekst omzet naar een 
		tekst van een bepaald aantal karakters. Een hash heeft de volgende 
		eigenschappen:
		\begin{itemize}
			\item Elke input geeft altijd dezelfde output
			\item Twee verschillende inputs geven nooit dezelfde output
			\item Het is onmogelijk om van de output, de input te bepalen
			\item Een kleine verandering in de input zorgt voor een grote 
			verandering in de output
		\end{itemize}
		Nooit betekent in dit geval onhaalbaar.
		Enkele voorbeelden hiervan zijn MD5, SHA1, SHA256 en SHA512. \\
		De letter \textit{H} zal in deze bachelorproef gebruikt worden als 
		notatie om een hash functie te gebruiken}
}

\newglossaryentry{toerekenbaarheid}
{
	name={toerekenbaarheid},
	description={
		Toerekenbaarheid wijst op een situatie waarin de auteur van een bericht 
		niet in staat is om te ontkennen dat hij of zij dit bericht 
		geschreven heeft. Engelse term: \textit{non-repudation}}
}

\newglossaryentry{integriteit}
{
	name={integriteit},
	description={
		Met integriteit wordt in dit document de ongewijzigde en
		consistente staat van data bedoeld \autocite{Boritz2005}. Met het
		garanderen van de integriteit wordt bedoeld dat een document
		of object niet ongemerkt gewijzigd kan worden}
}

\newglossaryentry{authenticiteit}
{
	name={authenticiteit},
	description={Authenticiteit bepaalt of de herkomst van data betrouwbaar is}
}

\newglossaryentry{TOR}
{
	name={TOR},
	description={
		TOR is gratis software dat anonimiteit op het internet belooft. TOR staat voor \textit{The Onion Router}. Dit is een verwijzing naar de    lagen structuur die TOR gebruikt tijdens het encrypteren en decrypteren van pakketten}
}

\newglossaryentry{OpenPGP}
{
	name={OpenPGP},
	description={
		OpenPGP is een standaard gedefinieerd door de OpenPGP Working Group of the Internet Engineering Task Force (IETF) in RFC 4880}
}


\newglossaryentry{GPG}
{
	name={GPG},
	description={
		GPG staat voor GNU Privacy Guard en is een gratis implementatie van de OpenPGP standaard}
}

\newacronym{NIST}{NIST}{National Institute of Standards and Technology}

\newacronym{PGP}{PGP}{Pretty Good Privacy}

\newacronym{PKI}{PKI}{Public Key Infrastructure}

\newacronym{CA}{CA}{Certificaatautoriteit}

\newacronym{WOT}{WOT}{Web Of Trust}
