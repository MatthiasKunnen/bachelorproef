\chapter{Web of trust}
\label{ch:web-of-trust}

Sinds \acrshort{CA}s niet altijd te vertrouwen zijn, ontstond uit de nood voor
een
betrouwbaar authenticatie systeem het concept van een web of trust, afgekort
\acrshort{WOT}. Dit systeem werkt met een gedecentraliseerd model waarbij niet wordt
vertrouwd op \acrshort{CA}s.

Het systeem werkt aan de hand van een soort sleutelhanger waarin alle sleutels
worden opgeslagen. Deze sleutelhanger wordt naar gerefereerd met het woord
‘ring’. Deze ring bevat publieke sleutels van entiteiten. In tegenstelling tot
een centrale autoriteit wordt hier enkel door de beheerder van de ring, sleutels
toegevoegd. De beheerder van de ring beslist ook in hoeverre sleutels vertrouwd
worden.

Een \acrshort{WOT} werd ontwikkeld voor \gls{PGP}. Wegens licentieproblemen
werd een nieuwe standaard ontwikkelt die \gls{OpenPGP} werd genoemd. GnuPG is een
gratis implementatie van de \gls{OpenPGP} standaard gemaakt door de Free Software
Foundation. In deze bachelorproef zal, wanneer gesproken wordt van een Web Of
Trust, uitgegaan worden van \gls{OpenPGP} als men spreekt over de standaard en over
\gls{GPG} als over een toepassing wordt gesproken.

Implementaties die conform zijn aan de \gls{OpenPGP} standaard bevatten een
vertrouwensschema. Een sleutel kan de volgende vertrouwensniveaus hebben:

\begin{itemize}
	\item Unknown

	Niets is bekend over het oordeel van de eigenaar van de sleutel in het
	ondertekenen van sleutels.

	\item None

	De eigenaar staat bekend om het incorrect ondertekenen van andere sleutels.

	\item Marginal

	De eigenaar verstaat wat de implicaties zijn van het ondertekenen van een
	sleutel en zal deze vooraf ondertekenen.

	\item Full

	De ondertekening van deze sleutel is even goed als je eigen ondertekening.
	\autocite{GNUManual}
\end{itemize}

Een andere term in de \gls{OpenPGP} standaard is geldigheid. Deze is verschillend van
vertrouwen. Enerzijds is er het vertrouwen in een eigenaar om andere sleutels te
valideren, anderzijds is er de geldigheid van een sleutel. De geldigheid van een
sleutel duidt op het al dan niet vertrouwen dat een sleutel in handen is van de
juiste identiteit.
Dit schema stelt een gebruiker in staat om de identiteit van een sleutel te
verifiëren zonder deze zelf te valideren maar zonder controle te verliezen van
wie deze rechten heeft. Een sleutel die ondertekend is door een sleutel die
volledig vertrouwen geniet zal geldig zijn. Een sleutel die ondertekent is door
drie marginaal vertrouwde sleutels zal ook geldig zijn. Dit aantal kan aangepast
worden zodat een \acrshort{WOT} flexibel kan zijn \autocite{GNUManualValidatingKeys}.

\section{Keyservers}
\label{sec:keyservers}

Het verspreiden van je publieke sleutel wordt natuurlijk best gedaan in persoon.
Wanneer het aantal personen waarmee je data uitwisselt toeneemt, wordt dit
moeilijker. Keyservers zijn servers waar je keys van kunt importeren en keys
naar kunt exporteren. Wanneer Alice Bob zijn sleutel ondertekent, dan zal Alice
de ondertekende sleutel exporteren naar een keyserver zodat anderen kunnen zien
dat Alice deze heeft ondertekend. Iedereen kan keys uploaden naar de keyservers
en deze doen zelf geen verificatie van de validiteit of betrouwbaarheid van
sleutels. Verificatie van sleutels is altijd de verantwoordelijkheid van degene
die een sleutel importeert. Er zijn vele keyservers bereikbaar en het merendeel
hiervan synchroniseert met elkaar, het maakt dus niet echt uit naar welke
keyserver je de key uploadt. Sinds keyservers enkel publieke sleutels bijhouden
en deze toch overal bekend mogen zijn, is het ook niet nodig om over een veilig
kanaal je sleutel te uploaden \autocite{GNUManualDistributingKeys}.

Keyservers staan niet toe om sleutels te verwijderen. Dit is om het
\acrshort{WOT} te
beschermen tegen ongeautoriseerde verwijderingen.

Keyservers hoeven niet per se publiek te zijn. Het is mogelijk om een keyserver
in het interne netwerk van een bedrijf te laten functioneren. Dit soort
keyserver wordt een private keyserver genoemd. Dergelijke private keyservers
bieden het voordeel dat ze niet publiekelijk doorzoekbaar zijn. Hierdoor is het
niet mogelijk om als buitenstaander de e-mailadressen van alle werknemers te
vinden en hierna e-mails te sturen naar e-mailadressen die niet bedoeld zijn om
extern benaderd te worden. Ook geeft dit meer vrijheid aan de beheerders van de
keyserver om sleutels te verwijderen zonder zich zorgen te maken over oude
sleutels die op servers staan waar zij geen toegang tot hebben.

\section{Terugtrekken van sleutels in een Web Of Trust}
\label{sec:terugtrekken-van-sleutels-in-een-wot}

Het verlies van een sleutel en het wachtwoord dat hieraan gelinkt is, heeft tot
gevolg dat eenieder die hier toegang tot heeft alle data kan decrypteren die
met de publieke sleutel is geëncrypteerd. Verder kunnen digitale handtekeningen
gemaakt worden. Ook bestaat de kans dat entiteiten blijven data encrypteren met
het publieke deel van de verloren sleutel die je niet meer kunt decrypteren
door een gebrek aan private sleutel.

\subsection{Revocation certificate}

Om dit probleem op te lossen bestaat het revocation certificate. Een revocation
certificate kan enkel gegenereerd worden met de private sleutel. Wanneer een
private sleutel verloren is, kan dit certificaat gepubliceerd worden waarmee
wordt tegengegaan dat data wordt geëncrypteerd met de teruggetrokken publieke
sleutel. Decryptie met de private sleutel en validatie van digitale
handtekeningen met de publieke sleutel blijft mogelijk.
\autocite{GNUManualGettingStarted}

Publicatie van een revocation certificate kan door het certificaat te importeren
in een sleutelring en de teruggetrokken sleutel te verzenden naar een keyserver.

\subsection{Designated revokers}
\label{subsec:designated-revokers}

Een andere mogelijkheid die in de \gls{OpenPGP} standaard aanwezig is, zijn designated
revokers. Een designated revoker is een bezitter van een sleutel die een
revocation certificate kan genereren en verspreiden van een andere sleutel.
\autocite{rfc4880} Om een sleutel aan te duiden als designated revoker is het
nodig om toegang te hebben tot de private sleutel en wachtwoord van de sleutel
aan wie de designated revoker zal worden toegewezen. Met andere woorden, het
toewijzen van een designated revoker kan enkel met toestemming van de eigenaar
van de sleutel. Meerdere sleutels kunnen aangewezen worden om de rol van
designated revoker te vervullen. Het toevoegen van een designated revoker is
onomkeerbaar. Dit zorgt voor een ideale mogelijkheid om sleutels terug te
trekken als derde partij zonder dat hiermee mogelijk is om digitale
handtekeningen te maken of berichten te encrypteren en decrypteren in de naam
van de eigenaar van de sleutel.

\subsection{Ontvangen van key revocations}
\label{subsec:ontvangen-van-key-revocations}

Als een persoon zijn revocation certificate uploadt op de keyservers betekend
dit nog niet dat elke correspondent deze herroeping ontvangt. Hiervoor moet de
eigenaar van een ring de sleutels controleren op updates. Dit kan gedaan worden
via het volgende commando \autocite{GnuPGFAQ}.

\begin{lstlisting}[language=bash]
$ gpg --refresh-keys
\end{lstlisting}

Men kan ook een daemon gebruiken zoals parcimonie om de sleutels op de ring te
updaten zonder aan de keyservers te tonen in welke sleutels je geïnteresseerd
bent. Dit systeem werkt via \gls{TOR}. Anonimiteit bij het gebruik van een Web Of
Trust valt echter buiten het domein van deze bachelorproef
\autocite{RiseUpOpenPGPBestPractices}.

\section{Wat kan dit oplossen}
\label{sec:wat-kan-dit-oplossen}

De Web Of Trust implementatie leidt tot het oplossen van \gls{authenticiteit},
\gls{integriteit} en \gls{toerekenbaarheid}. Een identiteit kan bewijzen wie hij of
zij is
door het ondertekenen van data doordat het bezit van de private sleutel en
wachtwoord de authenticatie bevestigt. \Gls{integriteit} van data kan bewezen
worden
door de handtekening van een bericht te valideren. Het ondertekenen van een
bericht of contract kan niet ontkend worden doordat een handtekening enkel
succesvol gevalideerd kan worden via een unieke sleutel en doordat de enige
persoon
met toegang tot de private sleutel en het wachtwoord hiervan, de eigenaar van de
sleutel is.

Specifiek zou dit het volgende oplossen in het voorbeeldverhaal. Wanneer Thomas
aan Jim beveelt om betaling te verrichten en hij later doorheeft dat hij zich
vergist heeft, kan hij niet ontkennen dat hij de betaling aangevraagd heeft. Dit
zuivert Jim van blaam. Hiervoor is echter vereist dat Thomas het bericht waarin
hij de betaling aanvraagt, ondertekent. Het is daarom ook best als alle
communicatie in het bedrijf verplicht standaard digitaal ondertekend is.

In het geval dat een impersonator van Thomas probeert om Jim te overtuigen om
een betaling te doen, zal Jim vragen voor een digitaal ondertekend bericht als
bevestiging. Een impersonator zal geen toegang hebben tot de private sleutel van
Thomas en dus hiermee geen digitale handtekening kunnen. Dit zou het einde
moeten zijn van deze soort fraude. Indien de impersonator toegang zou hebben tot
de private sleutel van Thomas en het wachtwoord hiervan, dan zijn er zeer grote
consequenties. Het veiligstellen van de sleutels en wachtwoord is een apart
hoofdstuk en wordt later op teruggekomen.

